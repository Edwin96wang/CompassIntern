\section{Introduction}
The COMPASS stands for "Common Muon and Proton Apparatus for Structure and Spectroscopy", a fixed target experiment for investigation of nucleon spin structure and hadron spectroscopy. The final experimental results are concluded by analyzing the data recorded by multiple detectors during scattering experiment. Due to the complexity and sensitivity of COMPASS detectors, recorded data can be easily sabotaged by unexpected external conditions, such as electronic malfunction or unusual shutdown of some components. The data with those unwanted effects should be selected out to improve the quality of data analysis in the final step. In this project, the abnormality resulting from these effects are only investigated for the data with different run numbers. By calculating and comparing values of multiple characteristic parameters of each run, the abnormal runs can be identified and further examined to postulate their probable causes. In the end, by checking the already existed information in log book, it then can be determined which runs should be discarded.