\section{Introduction}
COMPASS stands for "Common Muon Proton Apparatus for Structure and Spectroscopy", a fixed target experiment for investigation of nucleon spin structure and hadron spectroscopy. The final experimental results are concluded by analyzing the data recorded by multiple detectors during scattering experiment. Due to the complexity and sensitivity of the COMPASS detectors, recorded data can easily be sabotaged by unexpected external conditions, such as electronic malfunction or unusual shutdown of some components. The data with those unwanted effects should be selected out to improve the quality of any analysis based on this data. In this project, the abnormality of 2008 COMPASS data resulting from these effects are investigated on a run by run basis. By calculating and comparing values of multiple characteristic parameters of each run, the abnormal runs can be identified and further examined to determine their probable causes. In the end, by checking the COMPASS in log book, it then can be decided which runs should be discarded.
