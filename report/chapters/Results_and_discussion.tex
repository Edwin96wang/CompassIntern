
\section{Correlation and results}
As discussed in the last section, different characteristic parameters are investigated throughout runs. Except few cases such as long active time in the spill (see figure \ref{fig:EveN_spill_abnormal}) or double peaks from angular distribution of recoil proton (see figure \ref{fig:Recoil_proton_double_peak}), most of the abnormalities can be depicted by FWHM value of total invariant mass distribution (shown in figure \ref{fig:Total_mass_FWHM}). To decipher the cause behind all these phenomenon, correlations between the abnormalities of different parameters are inquired.

\subsection{Correlation of abnormalities}
The most two obvious parameters that can be correlated to the abnormality of total invariant mass distribution should be recoil proton and photon number reconstructed from both ECALs.
\subsubsection{Correlation with recoil proton}
The angular distribution of recoil proton is directly related to the scattering process. If spectrum and constitution of incoming particle beam are same for each run, the mass of unknown intermediate state $X$ can be reflected by the recoil proton due to four-momentum conservation. As is shown in subsection \ref{subsec:recoil_proton}, the width at lower half of angular distribution could be really large for abnormal runs. To establish the correlation, the value of width for recoil proton can plotted with the value of FWHM of invariant mass distribution for each run (see figure \ref{fig:Total_mass_recoil_proton}). At first sight, strong correlations appear for the run number 69612, 69816, 70650, 70654. When there is an increase on FWHM value, there is also an increase on the width from recoil proton. On the other hand, there are also some abnormal runs which are not correlated, such as run number 69811 and range 70223 $\sim$ 70240. However, for run number 69811, it is correlated with abnormality of three pions invariant mass (see figure \ref{fig:pion_mass}).


\subsubsection{Correlation with ECALs}

Comparing abnormal runs in figure \ref{fig:Total_mass_FWHM} and figure \ref{fig:Three_pion_mass_Graph}, one can easily notice that most of abnormal runs in total invariant mass distribution fail to show abnormality in invariant mass distribution of three pions. Thus, they must be shown by the parameters relating to photons. The parameter for photons is chosen to be the percentage of photons recorded by ECAL1 (denoted by $\alpha$) calculated by following equation:
\begin{equation}
\alpha = N_{ECAL1}/\left(N_{ECAL1}+N_{ECAL2}\right) \times 100\%
\end{equation}
where $N_{ECAL1}$ and $N_{ECAL2}$ is the number of photons reconstructed in ECAL1 and ECAL2 in same event respectively. Figure \ref{fig:Total_mass_ECAL} shows the comparison between photon percentage $\alpha$ and FWHM value of total invariant mass distribution. As can be seen clearly, excluding the correlation with recoil proton width, there are three regions where percentage $\alpha$ is negative correlated to the FWHM of total invariant mass. The increase of photon percentage $\alpha$ is accompanied by the slightly decrease of FWHM value. For those abnormal runs already correlated to recoil proton width or three pion invariant mass, the percentage $\alpha$ is positive correlated with the FWHM value. To investigate this negative correlation in further detail, the averaged number of photons per event from both ECALS are drawn with percentage $\alpha$ (see figure \ref{fig:ECAL_per_ECAL}). On all these negative correlation region, it can be seen that values of photon number from ECAL2 decline abruptly while photon number from ECAL1 is steady. However, when photon numbers from both ECALs decline or rise simultaneously, no abnormality occurs in FWHM value. Therefore, one can say that the FWHM of total invariant mass distribution is not correlated only to photon numbers from ECAL2, but rather from both calorimeters.
\subsection{Summary of abnormal runs}
During this study, 5 types of abnormal runs were found (see table \ref{tab:summary}) throughout the investigation of 4 characteristic parameters: event number, three pions mass, angular distribution of the recoil proton and number of photons reconstructed in ECALs. Possible causes of some abnormal runs can be directly checked out from logbook of COMPASS websites\footnote[2]{}. It can be found that irregular large values of recoil proton width (in figure \ref{fig:Recoil_proton_comp}) are mostly due to malfunction of triggers, especially sandwich vetos. This is reasonable since sandwich veto select out the events with large scattering angle. Due to four-momentum conservation, large scattering angles of pions correspond to small recoil angle of recoil proton. If sandwich veto fails to veto events of large scattering angle, the entries of small recoil angle in recoil proton angular distribution would be increased. For run number 69811, the three pion invariant mass distribution is shifted to small values, which could be successfully explained by increase of magnetic field. If the magnitude of the magnetic field rises, particles traveling through it would have a smaller radius of curvature according to following equation deduced by equality of Lorentz force and centripetal force:
\begin{equation}
p = q\cdot B\cdot r
\end{equation}
where $p$ is the momentum of the particle, $B$ is the magnitude of the magnetic field and $r$ is the radius of curvature. On the other hand, when calculating the momentum of particles from a smaller $r$, magnetic field is considered to be same, resulting in a smaller particle momentum. And this can be seen on figure \ref{fig:Three_pion_mass_69811}, where the second peak disappears and invariant mass of the three pions becomes smaller than a normal run.
\begin{table}[!t]
	\begin{tabular}{|p{0.1\textwidth}|p{0.15\textwidth}|p{0.19\textwidth}|}
		\hline
		RunNumber        & Abnormality                                      & Log book/comments                 \\ \hline \hline
		70195             & Disorder of spill time                                    & Good                              \\ \hline \hline
		69811             & Three pions mass                                           & Magnets were ramped up during run \\ \hline \hline
		70054             & Double peaks for recoiled proton                          & Errors appeared for SrcID         \\ \hline \hline
		69612             & \multirow{4}{0.15\textwidth}{Recoiled proton angular distribution}     & N/A                               \\ \cline{1-1} \cline{3-3} 
		69816             &                                                           & No sandwich veto                  \\ \cline{1-1} \cline{3-3} 
		70650             &                                                           & Special run to test sandwich veto \\ \cline{1-1} \cline{3-3} 
		70654             &                                                           & No sandwich veto                  \\ \hline \hline
		\parbox[c]{\hsize} {69686}             & \multirow{4}{0.15\textwidth}{Decrease of number of photons from ECAL2} & Detector test                     \\ \cline{1-1} \cline{3-3} 
		69687             &                                                           & Trigger problems            \\ \cline{1-1} \cline{3-3} 
		70223 $\sim$70240 &                                                           & High voltage trip on ECAL2        \\ \cline{1-1} \cline{3-3} 
		70448             &                                                           & Low intensity beam                \\ \hline
	\end{tabular}
	\caption{Summary of all abnormal runs throughout stability test. The first column shows run numbers with their corresponding abnormality described in second column. Log books or comments from COMPASS website of each run is shown in third column.}
	\label{tab:summary}
\end{table}